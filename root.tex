\documentclass[journal, final, letterpaper]{IEEEtran}

% packages
\usepackage{fixltx2e}
\usepackage{amsmath}
\usepackage[font=footnotesize,caption=false]{subfig}
\usepackage{mathtools}
\usepackage{graphicx}
\usepackage{url}
\usepackage{booktabs}
\usepackage[T1]{fontenc}
\usepackage{microtype}
\usepackage{tikz,pgfplots}
\usepackage{cite}
\usepackage[abs]{overpic}
\usepackage[applemac]{inputenc}
\usepackage{algorithmic}
\usepackage[caption=false,font=footnotesize]{subfig}
\usepackage{url}
\usepackage{siunitx}
\usepackage{cellspace}
\usepackage[nolist]{acronym}

% Mathematical Tool
\usepackage{amssymb,amsfonts,commath}
\usepackage[bb=boondox]{mathalfa}
% \usepackage{mathdots}
% \usepackage{etoolbox}
% \makeatletter
% \patchcmd{\@makecaption}
%   {\scshape}
%   {}
%   {}
%   {}
% \makeatletter
% \patchcmd{\@makecaption}
%   {\\}
%   {.\ }
%   {}
%   {}
%   \makeatother
% \def\tablename{Table}
% % circuits
% \usepackage[americanvoltages,siunitx]{circuitikz}


% % TIKZ package
% \usepackage{tikz}
% \usepackage{pgfplots}
% \pgfplotsset{every axis/.append style={line width=1pt}}
% \usetikzlibrary{shapes, fit, decorations.markings, matrix, plotmarks, positioning, fit, spy, patterns, shadows, calc, backgrounds, arrows}
% \usetikzlibrary{arrows.meta}
% \usepackage{tikzscale}
% \usepackage{subfig}

% \pgfplotsset{every tick label/.append style={font=\tiny}}

% \definecolor{mycolor1}{rgb}{0.00000,0.44700,0.74100}%
% \definecolor{mycolor2}{rgb}{0.85000,0.32500,0.09800}%
% \definecolor{mycolor3}{rgb}{0.92900,0.69400,0.12500}%
% \definecolor{mycolor4}{rgb}{0.49400,0.18400,0.55600}%
% \definecolor{mycolor5}{rgb}{0.46600,0.67400,0.18800}%
% \definecolor{mycolor6}{rgb}{0.30100,0.74500,0.93300}%
% \definecolor{mycolor7}{rgb}{0.63500,0.07800,0.18400}%

% %\pgfplotsset{every axis/.append style={
% %                    axis x line=middle,    % put the x axis in the middle
% %                    axis y line=middle,    % put the y axis in the middle
% %                    axis line style={<->}, % arrows on the axis
% %                    label style={font=\tiny},
% %                    tick label style={font=\tiny}  
% %                    }}

% % enumerate and equations
% \usepackage{enumerate}
% %\AtBeginEnvironment{align}{\setcounter{subeqn}{0}}% Reset subequation number at start of align
% \newcounter{subeqn} \renewcommand{\thesubeqn}{\theequation\alph{subeqn}}%
% \newcommand{\subeqn}{%
%   \refstepcounter{subeqn}% Step subequation number
%   \tag{\thesubeqn}% Label equation
% }


% theorems
\newtheorem{theorem}{Theorem}[section]
\newtheorem{corollary}{Corollary}[theorem]
\newtheorem{lemma}[theorem]{Lemma}
\newtheorem{proposition}{Proposition}[section]
\newtheorem{assumption}{Assumption}
\newtheorem{definition}{Definition}[section]
\newtheorem{alg}{Algorithm}[section]
\newtheorem{example}{Example}[section]
\newtheorem{remark}{Remark}
\newtheorem{problem}{Problem}

% Create shorthand definitions
\newcommand{\ts}[1]{{\textnormal{#1}}}
\newcommand{\diff}[2]{{\frac{d #1}{d #2}}}
\newcommand{\pdiff}[2]{{\frac{\partial #1}{\partial #2}}}
\newcommand{\ie}{\emph{i.e.}\ }
\newcommand{\eg}{\emph{e.g.}}
\newcommand{\cf}{\emph{c.f.}\ }
\newcommand{\Nset}{\mathbb{N}}
\newcommand{\Rset}{\mathbb{R}}
\newcommand{\Cset}{\mathbb{C}}
\newcommand{\mc}{\mathcal}
\newcommand{\mb}{\mathbf}
\newcommand{\mbb}{\mathbb}
\newcommand{\mf}{\mathfrak}
\newcommand{\ms}{\mathscr}
\newcommand\st{\ \colon\ }


% Activate selected proofs
\newif\ifproves
\provestrue

\allowdisplaybreaks

\begin{acronym}
	\acro{MPC}{Model Predictive Control}
	\acro{PV}{Photo-voltaic}
	\acro{DG}{Distributed Generation}
	\acro{DS}{Distributed Storage}
	\acro{DER}{Distributed Energy Resources}
	\acro{BIC}{Bounded Integral Controller}
	\acro{VCS}{Voltage controlled source}
	\acro{MG}{Micro-grid}
	\acro{OPF}{Optimal Power Flow}
	\acro{SoC}{State of Charge}
	\acro{WT}{Wind Turbine}
	\acro{PCC}{Point of Common Coupling}
	\acro{OCP}{Optimal Control Problem}
	\acro{DAS}{Differential Algebraic System}
	\acro{DAE}{Differential Algebraic Equation} 
\end{acronym}
%
\interdisplaylinepenalty=2500

% BiBTeX
\bibliographystyle{IEEEtran}

%



\makeatletter
\renewcommand*\env@matrix[1][c]{\hskip -\arraycolsep
  \let\@ifnextchar\new@ifnextchar
  \array{*\c@MaxMatrixCols #1}}
\makeatother


\begin{document}

\title{Distributed control of Hybrid microgrids}

% Author names
\author{Pablo R. Baldivieso-Monasterios \IEEEmembership{Member,
      IEEE}, Andrei Braitor \IEEEmembership{Member,
      IEEE}, George C. Konstantopoulos, \IEEEmembership{Member,
      IEEE}
\thanks{P.~R.~Baldivieso-Monasterios, Andrei Braitor, and G.~C.~Konstantopoulos are with the Department of Automatic Control \& Systems Engineering,
      University of Sheffield, Mappin Street, Sheffield S1 3JD, UK
      (e-mail: \texttt{\{p.baldivieso,a.braitor, g.konstantopoulos\}@shef.ac.uk}).}}

%\markboth{IEEE Transactions on Energy Conversion}{Baldivieso-Monasterios \& Konstantopoulos \& Alexandridis: Model based advanced energy conversion in $\mu\ts{Grids}$}

\maketitle

\begin{abstract}
Many things happen here
\end{abstract}
%
\begin{IEEEkeywords}
Micro-grids, Droop free control, Optimal control, secondary control
\end{IEEEkeywords}
%
\section{Introduction}
\label{sec:introduction}

\underline{\emph{Notation}}: The $n-$dimensional \emph{sphere} and \emph{torus} are denoted as $\mbb{S}^n$ and $\mbb{T}^{n}$ respectively. The identity matrix in $\Rset^{n\times n}$ is $\mbb{I}_n$, similarly a complex structure in $\Rset^{2n}$ is \[\mbb{J} =
  \begin{bmatrix*}[r]
    0~ &-\mbb{I}_n \\
    \mbb{I}_n & 0~ 
  \end{bmatrix*}
  .\]A $n\times m$ matrix with only ones and zeros entries are denoted $\mbb{1}_{n\times m}$ and $\mbb{0}_{n\times m}$ respectively; subindices are omitted when dimensions can be understood from context.

Hi Andrei

\begin{definition}
The system $\dot{x} = f(x,u)$ with outputs $y = h(x)$ is passive if there exists a $C^{1}$ function $V:\Rset^n\to\Rset_{\geq0}$ such that
%
\begin{equation*}
 \dot{V} = \frac{\partial V}{\partial x}f(x,u) \leq y^\top u,\quad \forall (x,u)\in\Rset^{n}\times\Rset^{m}.
\end{equation*}
% 
\label{def:passivity}	
\end{definition}

%%
%%

%
%
\section{\ac{MG} modelling}
\label{sec:mg_modelling}
%%
%%

\begin{assumption}[AC \ac{MG} interconnection] The \ac{MG} components satisfy:
\begin{enumerate}[i)]
	\item The inverters, lines, and loads are connected to each other via a $Y$ topology.
	\item The graph $\mc{G} = (\mc{V},\mc{E})$ has only one connected component. 
\end{enumerate}
\label{asssump:MG_connected}
\end{assumption}
%
The network of interlinking DC/AC converters obeys the following differential-algebraic dynamics for $\mu\in\mc{V}$
%
\begin{subequations}
  \begin{align}
    C_\mu \diff{V_\mu}{t} & = -\frac{1}{R_\mu}V_\mu + g_\mu(P_\mu^\ts{in},V_\mu) - \frac{1}{2}m_\mu^\top i_\mu\\
    L_{f,\mu} \diff{i_{f,\mu}}{t} &= -Z_{f,\mu}(\omega_\mu) i_{f,\mu} - v_{f,\mu} + \frac{1}{2}V_\mu m_\mu\\
    C_{f,\mu} \diff{v_{f,\mu}}{t} &= -Y_{f,\mu}(\omega_\mu) v_{f,\mu} + i_{f,\mu} - i_{\mu}\\
    L_\mu \diff{i_{\mu}}{t} &= -Z_\mu(\omega_\mu) i_{\mu} + v_{f,\mu} -T_{dq}(\alpha_\mu)v_{\mu}\\
    \diff{\alpha_\mu}{t}  &= \omega_\mu\\
    L_\mc{E}\diff{i_\mc{E}}{t} & = -r_\mc{E}i_\mc{E} + \mc{B}^\top\otimes\mbb{I}_3 v\\
    0  &= T^{-1}(\alpha_\mu)i_\mu - \mc{B}_\mu\otimes\mbb{I}_3 i_\mc{E} - i_{L,\mu}\label{eq:alg_eq}
\end{align}
\label{eq:mg_AC_model}
\end{subequations}
%
In this model, the connecting voltages $v = (v_1,\ldots,v_M)\in\Rset^{3M}$ represent algebraic variables; note that these do not appear directly in the algebraic equation~\eqref{eq:alg_eq} but appear in its derivative which makes the system at least of index 1. The differential variables contain both local variables $(V_\mu,i_{f,\mu},v_{f,\mu},i_\mu,\alpha_\mu)\in\Rset^{7}\times\mbb{S}^{1}$ for each $\mu\in\mc{V}$ and network variables $i_\mc{E}\in\Rset^{3M_\mc{E}}$ which correspond to DC voltage, filter currents, filter voltages, connection currents, and line currents respectively. The current drawn by each load $i_{L,\mu}$ appears in the algebraic constraint which relates the input from a source $T_{dq}^{-1}i_{\mu}$ and the network contribution $\mc{B}_{\mu}\otimes\mbb{I}_3i_\mc{E}$ where $\mc{B}_\mu$ correspond to the node-edge incidence matrix  $\mu^\ts{th}$ row. The controllable inputs are given by $u_\mu = (m_u,\omega_\mu)$ for each $\mu\in\mc{V}$ and correspond to each converter modulation indices and frequencies at each node. The exogenous inputs are given the load consumption $i_L\in\Rset^{3M}$, and renewable input power $P = (P_1,\ldots,P_M) \in \Rset^{M}$.


In this paper we assume that each \ac{DER} is in closed-loop with their respective primary controllers [CITE PAPERS FROM THE GROUP], as a result the current injected by each \ac{DER} $i\in\mc{V}$ can be expressed as a controlled $3-$phase current source
%
\begin{subequations}
\begin{align}
  \diff{\alpha_i}{t} =& \omega_i,\label{eq:current_der_dyn}\\
  i_i = & \xi_i(\alpha)I_i 
\end{align}
\label{eq:current_der}	
\end{subequations}
% 
 The controlled variables are the amplitude and frequency $u_i = (I_i,\omega_i)$; for a constant frequency, the resulting angle is a linear function on the circle, \ie $\alpha_i(t) = \mod_{2\pi}(\alpha_i(0)+\omega_it)$. The collection of all angles $\alpha = (\alpha_i)_{i\in\mc{V}}$ are defined in the $M-$dimensional \emph{torus} $\mbb{T}^M$; whereas the collection of amplitudes and frequencies are $I = (I_i)_{i\in\mc{V}}$ and $\omega = (\omega_i)_{i\in\mc{V}}$ are defined on $\Rset^{2M}$. Interconnecting \eqref{eq:current_der} and~\eqref{eq:line_node_interconnection} yields the overall model 
%
\begin{subequations}
\begin{align}
  \begin{bmatrix}	\diff{\tilde{x}}{t}\\\diff{\alpha}{t}  \end{bmatrix} &= \begin{bmatrix} J-R & \mbb{0}\\\mbb{0} & \mbb{0}\end{bmatrix} \begin{bmatrix} \pdiff{H}{\tilde{x}}\\\pdiff{H}{\alpha}\end{bmatrix} +G(\alpha)\begin{bmatrix}I\\\omega\\ i_L\end{bmatrix}\\
  y &= G(\alpha)^\top\begin{bmatrix} \pdiff{H}{\tilde{x}}\\\pdiff{H}{\alpha}\end{bmatrix}.
\end{align}
\label{eq:MG_model}	
\end{subequations}
%
 The resulting inputs controllable inputs are $u = (I,\omega)$ which corresponds to amplitudes and frequencies from each \ac{DER}. As mentioned lines above, the remaining inputs are used to interface the \ac{MG} to local loads. The input matrix is state dependent
 \[
 G(\alpha) = \begin{bmatrix} \xi(\alpha) & \mbb{0}_{3M\times M} & -\mbb{I}_{3M}\\ \mbb{0}_{3M_E\times M} & \mbb{0}_{3M_E\times M} & \mbb{0}_{3M_E\times 3M}\\\mbb{0}_{M\times M} & \mbb{I}_{M} & \mbb{0}_{M\times 3M}\end{bmatrix},
 \]
 %
and its associated outputs are $y = (\xi(\alpha)^\top\pdiff{H}{q},\pdiff{H}{\alpha},-\pdiff{H}{q})$. The modified Hamiltonian for model~\eqref{eq:MG_model} incorporates an extra term related to the angle
%
\begin{equation}
  H = \frac{1}{2}\bigl ( q^\top C^{-1}q +\varphi^\top L^{-1}\varphi  \bigr ) + h^\top(\mbb{1} - \cos\alpha)
\label{eq:Hamiltonian_MG}  
\end{equation}
%
 where $h\in\Rset^M$ is a vector containing energy coefficients. The resulting state vector $x = (\tilde{x},\alpha)$ is defined on the set $\mc{X} = \Rset^{3(M+M_E)}\times\mbb{T}^M$. Furthermore, for constant inputs, the system is time-varying, the sinusoidal characteristics of the input map propagates through the linear dynamics which implies that the states $x\in\Rset ^{3(M+M_E)}$ have sinusoidal behaviour. As a result, we can define amplitudes, angles, and frequencies at each node via the operators 
 %
\begin{equation}
 \theta_i = \ts{Ang}(q_i)\in\mbb{S}^1,\quad V_i = C_i^{-1}\ts{Rms}(q_i)\in\Rset.
\label{eq:angle_rms}  
\end{equation}
%
and the relation 
%
\begin{equation}
	\theta_i(t) = \theta_i(0) + \int_0^t\diff{\theta_i}{\tau}(\tau)d\tau 
\label{eq:angle}  
\end{equation}
%
 The resulting nonlinear system has its nonlinearity appearing through the input
 

 
 In open loop format, this system is of non-minimum phase 
%
\subsection{Control objectives}
\label{sec:control_objectives}
%


The controller objectives are twofold: achieve frequency synchronisation and to keep the voltages within a prescribed range. 

A precise definition of synchronisation can be found in  \cite{Dorfler2014}:
 %
 \begin{definition}
A solution $ \theta:\Rset_+\to \mbb{T}^M$ achieves \emph{Frequency Synchronisation} if all frequencies $\diff{ \theta_i}{t}(t)$ converge to a common constant frequency $\omega^\ts{sync}\in\Rset$ as $t\to\infty$.
 \label{def:freq_sync}	
 \end{definition}
%
\begin{definition}
For any $\gamma\in[0,\pi)$ and a graph $\mc{G} = (\mc{V},\mc{E})$, let $\bar{\Delta}_{\mc{G}}(\gamma) = \{ \theta\colon | \theta_i -  \theta_j|\leq\gamma,~i:j\in\mc{E}\}\subset\mbb{T}^M$. A solution $\alpha:\Rset_+\to \mbb{T}^M$ is \emph{Phase Cohesive}  if there exists $\gamma\in[0,\pi)$ such that $ \theta(t)\in\bar{\Delta}_\mc{G}$ for all $t\geq 0$.
\label{def:phase_cohes}	
\end{definition}

%



Transforming the system into a rotating frame with constant angular velocity $\omega^\ts{com}$, the quantity 
\[
\delta_i(t) = \alpha_i(0) + \int_0^t(\omega_i(\tau) -\omega^\ts{com})d\tau 
\]
measures the angle deviation from node $i$ with respect to that of the reference frame. Following definition 
\cite{Simpson-Porco2013}


% 
The resulting \ac{DAS} can be defined as






when new elements are connected to the network, each inverter connects to the network via a model 

This section presents the model of the \ac{MG} 
%
\begin{subequations}
\begin{align}
C\diff{v}{t} =& -(Y+L_E) v + g(\delta)i_d - i_L,\label{eq:dyn_global_voltages}\\
T\diff{\omega}{t} =& -\omega+\omega^\ts{d} + M_\phi \sin\mb{\phi},\label{eq:dyn_global_frequency}\\
M\diff{\phi}{t} =& K_i[\cos\phi]L_C\omega,\label{eq:dyn_global_bic}\\
\diff{\delta}{t} =& \omega - \mbb{1}\omega^\ts{com},\label{eq:dyn_global_angle}
\end{align}
\label{eq:dyn_global}	
\end{subequations}
%
This model captures the voltage, frequency, and angle dynamics, as well as the BIC controller used to promote frequency synchronisation. The overall state is $x = \{(v_{d,i},v_{q,i},\omega_i,\delta_i,\phi_i)\}_{i\in\mc{V}}$ which is the aggregation of the states from all the members of the network. The topology of both physical and control networks is represented through the Laplacian matrices $L $ and $L_{C}$; the former encapsulates the information on the dynamical interconnections, \ie the admittance matrix, whereas the latter is used as a mechanism to promote frequency synchronisation. The inputs of the \ac{MG} are the current injections from each node $u = \{i_{d,i},\omega^{d}_{i}\}_{i\in\mc{V}}$.

The model is expressed in the global $dq0$ framework rotating at a constant speed $\omega^c$. The synchronisation frequency is $\omega_c$ which may not be equal to the desired frequency $\omega_d$. In this paper, the primary controller acts on both the current frequency and its 


The \ac{MG} synchronised motion can be defined as all the nodes oscillating at the same constant frequency $\omega^*\in\Rset$ such that $\alpha_i(t) =\omega^*t$ and a constant voltage amplitude for all $i\in\mc{V}$. As a result, and from following~\eqref{eq:dyn_global_angle}, the resulting angle satisfies $\delta(t) = t(\omega^* - \omega^\ts{com})\mbb{1} + \alpha(0)$ and if the common $dq0$ frequency is the synchronised one, then $\delta(t) = \alpha(0)$. A direct consequence of setting $\omega = \omega^*\mbb{1}$, is that $\omega\in\ker{L_C}$, which implies $\phi$ achieving an equilibrium value $\phi^{e}\in\mbb{T}^{M}$. Furthermore, $\diff{\omega}{t} = 0$ yielding
%
\begin{align*}
	M\omega^* &= \mbb{1}^\top\omega^{d} + M_\phi\mbb{1}^\top\sin\phi\\
	\omega^* & \leq  \frac{1}{M} \mbb{1}^\top\omega^{d} + M_\phi
\end{align*}
%
The synchronisation frequency is bounded from above by the frequency set-points. On the other hand, the current inputs are determined by the desired voltage:
%
\[
i_{d} = g(\alpha(0))^\top((Y+L ) v + h_L(v))
\]
where $h_L(v)$ characterises the, possibly nonlinear, relationship between load voltages and currents. 


\begin{assumption}[Balanced \ac{MG}]
The \ac{MG} is a balanced power system: for all $i\in\mc{V}$, the voltage and current associated to that node satisfy $\mbb{1}_3^\top v_i = 0$ and $\mbb{1}_3^\top i_i = 0$.
\label{asssump:MG_balanced}
\end{assumption}

\begin{lemma}[Power computations in the $abc$ framework]
	Suppose Assumption~\ref{asssump:MG_balanced} holds. The $3$-phase active and reactive power are given by\footnote{$\times$ is the standard cross product in $3$ dimensions.}
	\begin{equation}
		P = v^\top i;\quad Q = \frac{\sqrt{3}}{3}(\mbb{1}_3\times v)^\top i.
	\end{equation}
	for any $3$-phase voltage $v$ and current $i$. 
\label{lem:power_computation}
\end{lemma}

The current demanded by the load, $i_{l}\in\Rset^{3}$, depends on the active power, reactive power $S_l = (P_l,Q_l)$
%
\begin{equation}
  	C_{ll}\frac{dv_{l}}{dt}= -G_{ll} v_{l} - h_l(v_l,S_l) + B_l^\top\mb{i}_{E},
\label{eq:load_voltage}
\end{equation}
%
and $h_l\colon\Rset^{3}\times\Rset^{2}\rightarrow\Rset^{3}$ is a function of load power and voltage.
%
\begin{lemma}
	Suppose Assumption~\ref{asssump:MG_balanced} holds. For constant power loads, $S_l=(P_l,Q_l)\in\Rset^{2}$, the mapping $v\mapsto h(v,S_l)$ is well defined for all $v\in\mbb{V}= \{v\in\Rset^3\colon \ts{Rms}(v) \neq 0\}$ where
	%
	\begin{equation}
		h_l(v_l) = \frac{1}{\ts{Rms}(v_l)}T_\ts{dq}^\top(\ts{Ang}(v_l)).
	\label{eq:cpl_current}  
	\end{equation}
	%
	\label{lem:cpl_current}


\end{lemma}

\bibliography{extracted}

%\newcommand{\myreferences}{/Users/Pablo/Dropbox/Reports/library}
%\newcommand{\myreferences}{/Users/prbalm/Dropbox/Reports/library.bib}
%\newcommand{\myreferences}{library}
%\newcommand{\myreferences}{}
\appendix
%
%
\subsection{Proof of Lemma~\ref{lem:power_computation}}
\label{sec:proof1}
%
%
Given the voltage and current in the $abc$ framework, \ie $v_\ts{abc},i_\ts{abc}\in\Rset^3$ with angles $\alpha(t)$ for the voltage and $\beta(t) = \alpha(t) + \delta(t)$ for the current, and the time varying coordinate transformation $T_\ts{dq}(\sigma(t))$. The expressions of these in the $\ts{dq0}$ framework are $v_\ts{dq} = (v_d,v_q,0) = T_\ts{dq}(\sigma(t))v_\ts{abc}$ and $i_\ts{dq} = (i_d,i_q,0)=T_\ts{dq}(\sigma(t))i_\ts{abc}$ when the system is balanced. The corresponding power expressions are $P_i = \frac{3}{2}(v_di_d + v_qi_q)$ for the active power and $Q_i = \frac{3}{2}(v_qi_d - v_di_q)$. Expressing these in matrix form:
%
\begin{align*}
	P = \frac{3}{2}v_\ts{dq}^\top i_\ts{dq}; && Q = \frac{3}{2}v_\ts{dq}^\top\underbrace{\begin{bmatrix}0 &-1 & 0\\1 & 0 & 0\\0& 0& 1\end{bmatrix}}_{\Xi^*} i_\ts{dq}
\end{align*}
%
The power, active and reactive, is invariant under coordinate transformations $P_\ts{abc} = P_\ts{dq}$ and $Q_\ts{abc} = P_\ts{dq}$. 
%
\begin{align*}
	P &= \frac{3}{2}v_\ts{abc}^\top T_\ts{dq}(\sigma(t))^\top T_\ts{dq}(\sigma(t)) i_\ts{abc};\\  
	Q &= \frac{3}{2}v_\ts{dq}^\top T_\ts{dq}(\sigma(t))^\top\Xi^*T_\ts{dq}(\sigma(t)) i_\ts{abc}
\end{align*}
%
Since $T_\ts{dq}(\sigma(t))$ satisfies, $\frac{3}{2}T_\ts{dq}(\sigma(t))^\top T_\ts{dq}(\sigma(t)) = \mb{I}_3 + k\mbb{1}_{3\times 3}$ and $\frac{3}{2}T_\ts{dq}(\sigma(t))^\top \Xi^*T_\ts{dq}(\sigma(t)) = \frac{\sqrt{3}}{3}\Xi + k\mbb{1}_{3\times 3}$ with $\Xi$ skew symmetric such that
%
\begin{equation*}
  \Xi = \begin{bmatrix} 0&-1 & 1\\ 1& 0& -1\\ -1 &-1 &0\end{bmatrix}
\end{equation*}
% 
This matrix defines the vector product in $3$ dimensions with the unity vector , \ie $i_\ts{abc}\times\mbb{1}_3 = \Xi i_\ts{abc}$. Because the system is balanced, we obtain $\mbb{1}_{3\times 3}i_\ts{abc} = \mb{0}$, resulting in the expressions for the active and reactive power
%
\begin{align*}
	P = v_\ts{abc}^\top i_\ts{abc}; && Q = \frac{\sqrt{3}}{3}v_\ts{abc}^\top(i_\ts{abc}\times\mbb{1}_{3})
\end{align*}
%
Using the properties of the triple product yield the desired expressions
\begin{align*}
	P = v_\ts{abc}^\top i_\ts{abc}; && Q = \frac{\sqrt{3}}{3} (\mbb{1}_{3} \times v_\ts{abc})^\top i_\ts{abc}
\end{align*}
%
\subsection{Proof of Lemma~\ref{lem:cpl_current}}
\label{sec:proof2}
%
%
Given constant values for the active and reactive power $(P,Q)\in\Rset^{2}$, from Assumption~\ref{asssump:MG_balanced} and Lemma~\ref{lem:power_computation}, the powers can be expressed in matrix form as follows
%
\begin{equation*}
  \begin{bmatrix}P\\Q\\0\end{bmatrix} = \underbrace{\begin{bmatrix}v_\ts{abc}^\top\\\frac{\sqrt{3}}{3} (\mbb{1}_{3} \times v_\ts{abc})^\top\\\frac{1}{2}\mbb{1}_3^\top\end{bmatrix}}_{\tilde{h}(v_\ts{abc})} i_\ts{abc}
\end{equation*}
%
The transformation $\tilde{h}(v_l)$ maps the $3-$phase currents to $\Rset^{3}$. This linear transformation is invertible for all $v_l\in\mbb{V}_{l}$. Indeed, for any $v_l\in\mbb{V}_l$, the RMS value of the voltage does not vanish, $V^\ts{Rms}(v_l)\neq 0$. In addition, the linear map is a matrix function $\tilde{h}:\Rset\rightarrow\Rset^{3\times 3}$ whose determinant satisfies $\det(\tilde{h}(v_l)) = 3\sqrt{3}V^\ts{Rms}(v_l)\neq 0$; as a result, the map is invertible for all $v_l\in\mbb{V}_l$. 

Following the definition of the $dq$ transformation of section~\ref{sec:MG_model:freq_sync}, and exploiting the structure of the power calculations, the constant power transform can be written as $\tilde{h}(v_l) = {V^\ts{Rms}(v_l)} T_\ts{dq}(\delta(v_l))$ yielding the desired expression  
\begin{equation*}
		h_l(v_l) = \tilde{h}^{-1}(v_l) = \frac{1}{V^\ts{Rms}(v_l)}T_\ts{dq}^\top(\delta(v_l)).
\end{equation*}


\end{document}
%%% Local Variables:
%%% mode: latex
%%% TeX-master: t
%%% End:
